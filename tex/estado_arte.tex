% Contenidos del capítulo.
% Las secciones presentadas son orientativas y no representan
% necesariamente la organización que debe tener este capítulo.

\section{Análisis de aplicaciones similares}
% Qué aplicaciones similares hay y en qué se diferencia de ellas la propuesta
Cada vez es mayor el peso que la tecnología esta teniendo en el día a día de la sociedad. La interacción con dispositivos tecnológicos está convirtiéndose poco a poco en algo cotidiano. Son cada vez mas las tareas que pueden ser suplidas por dispositivos electrónicos que ayudan a que tareas del día a día, incluso labores más peligrosas puedan desarrollarse de una forma segura y eficiente.

Uno de los sectores en auge en esta última década es el sector de los sistemas embebidos y la capacidad de procesamiento de estos mismos, es cada vez mayor la capacidad de dotar de inteligencia a dispositivos cada vez con dimensiones mas reducidas pero con una capacidad de rendimiento que crece de forma exponencial.

El proyecto que se describirá a lo largo de la memoria se encarga de dotar de inteligencia artificial un robot con capacidad de teleoperación y reconocimiento del entorno en el cual se esté operando este mismo. Para poder conseguir esto, se ha tenido que poner a punto un robot que sea capaz de moverse y además sea capaz de procesar y reconocer los objetos de su entorno procesando esta información en tiempo real.

Para conseguir esto se ha dotado al robot de visión artificial, la cual recogerá la información necesaria de una cámara incorporada en el robot y que se complementará con otros sensores que ayudaran al dispositivo móvil a poder moverse de una forma autónoma evitando colisiones.
\section{Tecnologías}
% Análisis crítico de las tecnologías y sistemas de despliegue posibles y por qué se han seleccionado unas concretas.

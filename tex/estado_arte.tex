% Contenidos del capítulo.
% Las secciones presentadas son orientativas y no representan
% necesariamente la organización que debe tener este capítulo.

\section{Tecnologías utilizadas}
% Qué aplicaciones similares hay y en qué se diferencia de ellas la propuesta
Cada vez es mayor el peso que la tecnología esta teniendo en el día a día de la sociedad. La interacción con dispositivos tecnológicos está convirtiéndose poco a poco en algo cotidiano. Son cada vez mas las tareas que pueden ser suplidas por dispositivos electrónicos que ayudan a que tareas del día a día, incluso labores más peligrosas puedan desarrollarse de una forma segura y eficiente.

Uno de los sectores en auge en esta última década es el sector de los sistemas embebidos y la capacidad de procesamiento de estos mismos, es cada vez mayor la capacidad de dotar de inteligencia a dispositivos cada vez con dimensiones mas reducidas pero con una capacidad de rendimiento que crece de forma exponencial.

El proyecto que se describirá a lo largo de la memoria se encarga de dotar de inteligencia artificial un robot con capacidad de teleoperación y reconocimiento del entorno en el cual se esté operando este mismo. Para poder conseguir esto, se ha tenido que poner a punto un robot que sea capaz de moverse y además sea capaz de procesar y reconocer los objetos de su entorno procesando esta información en tiempo real.

Para conseguir esto se ha dotado al robot de visión artificial, la cual recogerá la información necesaria de una cámara incorporada en el robot y que se complementará con otros sensores que ayudaran al dispositivo móvil a poder moverse de una forma autónoma evitando colisiones.
\section{Estudio previo}
El proyecto se apoya en el microcontrolador ESP32, el protocolo HTTP/HTTPS, el servidor en Python con Flask y una variedad de bibliotecas adicionales para ofrecer una solución completa y eficiente. Estas tecnologías permiten la captura de imágenes, la comunicación segura con el servidor y el procesamiento de los datos recibidos. La elección de estas tecnologías se basa en su adecuación al propósito del proyecto, su capacidad de cumplir con los requisitos de rendimiento y seguridad, así como su popularidad y soporte en la comunidad de desarrollo.

\subsection{Microcontrolador ESP32}
El microcontrolador ESP32, desarrollado por Espressif Systems, es el componente central de este proyecto. Este dispositivo se destaca por su alto rendimiento, bajo consumo de energía y amplias capacidades de conectividad. Cuenta con un procesador de doble núcleo, conectividad Wi-Fi y Bluetooth, así como interfaces GPIO para interactuar con diversos periféricos. El ESP32 ofrece una plataforma sólida y versátil para implementar aplicaciones IoT y sistemas embebidos.

%% Aqui meter una comparativa de porquie he escogido el modulito que he escogido

\subsection{Servidor HTTP Python}
Para la comunicación entre el microcontrolador y el servidor, se ha utilizado el protocolo HTTP (Hypertext Transfer Protocol) y su versión segura HTTPS (HTTP Secure). Estos protocolos permiten la transferencia de datos entre el cliente (microcontrolador) y el servidor de forma segura y eficiente. El uso de HTTPS garantiza la encriptación de los datos transmitidos, protegiendo la confidencialidad y la integridad de la información.

\subsubsection{Entorno Linux}
Existen diferentes recursos que se pueden utilizar a la hora de poder levantar un servidor HTTP, asi como diferente hardware en el que poder hacerlo. En el proyecto que se describe, la intención es dotar al proyecto de la mayor flexibilidad posible, la plataforma elegida ha sido un entorno Linux.

Linux es conocido como un sistema operativo basado en código abierto y gratuito, dicho sistema operativo se sostiene sobre un kernel, donde reside la verdadera funcionalidad de Linux, por tanto, una mejor descripción sería decir que, Linux, es un kernel sobre el cual se basa un sistema operativo. Fue creado por Linus Torvalds en 1991 como un proyecto personal y se ha convertido en uno de los sistemas más populares y ampliamente utilizados en el mundo.

Gran parte de la popularidad y del auge en el uso de Linux, es nu naturaleza de código abierto, más comunmente conocido com "open source", lo que significa que el código fuente del kernel está disponible para que cualquier persona lo pueda ver, modificar y distribuir. Esto fomenta que la comunidad de Linux sea completamente abierta, por lo que favorece la mejora de su funcionamiento asi como al añadido continuo de nuevas funcionalidades, las cuales poco a poco abarcan más campos de desarrollo.

Otra de las razones por las cuales Linux es tan utilizado en la industria, reside en, la capacidad del adaptación del kernel de Linux a cualquier arquitectura hardware del dispositvo que se quiera poner en marcha para un proyecto. De hecho, es muy comun utilizar hardware ligeramente diferente en las etapas de prototipado y en la puesta en marcha en producción. Por lo cual, se puede utilizar una misma version de kernel para diferentes arquitecturas y el código que se escriba no difiere de un hardware a otro, o si lo hace, son cambios superficiales. Esto ocasiona que, podamos utilizar la misma version del Kernel de Linux tanto en el PC en el que estemos desarrollando, así comoen el harware el cual pondremo en funcionamiento.


\begin{table}[h]
\centering
\caption{Comparación de servidor en Flask con otros lenguajes}
\label{tabla:comparacion-servidor}
\begin{tabular}{|l|l|l|}
\hline
\textbf{Lenguaje}    & \textbf{Complejidad de programación} & \textbf{Consumo de recursos}  \\ \hline
Python (Flask)       & Baja                                 & Moderado                      \\ \hline
Python (MongoDB)     & Baja a moderada                      & Moderado                      \\ \hline
Node.js              & Baja a moderada                      & Moderado a alto               \\ \hline
Java (Spring)        & Moderada                             & Alto                          \\ \hline
Ruby (Ruby on Rails) & Moderada                             & Moderado a alto               \\ \hline
C/C++ (Laravel)      & Moderada a alta                      & Bajo                          \\ \hline
\end{tabular}
\end{table}

%% Aqui meter una comparativa de porquie he elegido hacer el servidor en flask y no por ejemplo en otro lenguaje

\subsection{Módulos y librerias adicionales}
Además de las tecnologías mencionadas, se han utilizado diversas bibliotecas y módulos adicionales para la implementación y funcionalidad del sistema. Estas incluyen bibliotecas para el manejo de imágenes, como PIL (Python Imaging Library), para el procesamiento y análisis de datos, como NumPy y OpenCV, y para la encriptación y autenticación en el lado del servidor, como OpenSSL.

% Análisis crítico de las tecnologías y sistemas de despliegue posibles y por qué se han seleccionado unas concretas.

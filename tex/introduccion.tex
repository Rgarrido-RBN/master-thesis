% Contenidos del capítulo.
% Las secciones presentadas son orientativas y no representan
% necesariamente la organización que debe tener este capítulo.

\section{Introducción}
Cada vez es mayor el peso que la tecnología esta teniendo en el día a día de la sociedad. La interacción con dispositivos tecnológicos está convirtiéndose poco a poco en algo cotidiano. Son cada vez mas las tareas que pueden ser suplidas por dispositivos electrónicos que ayudan a que tareas del día a día, incluso labores más peligrosas puedan desarrollarse de una forma segura y eficiente.

Uno de los sectores en auge en esta última década es el sector de los sistemas embarcados y la capacidad de procesamiento de estos mismos, es cada vez mayor la capacidad de dotar de inteligencia a dispositivos cada vez con dimensiones mas reducidas pero con una capacidad de rendimiento que crece de forma exponencial.

El proyecto que se describirá a lo largo de la memoria se encarga de dotar a determinados dispositivos de la capacidad de comunicación con un servidor centrar, utilizado para procesar una determinada informacion que cada uno de los dispotivivo decida enviar, generalmente, sistemas de visión. Para poder conseguir esto, se ha tenido que poner a punto microcontrolador capaz de poder capturar imágenes a través de una cámara. Posteriormente, en un servidor central, dicha información será atendida y posteriormente procesada para extraer la información necesaria.

\section{Objetivos}
El objetivo principal de este proyecto es establecer una comunicación bidireccional efectiva entre el microcontrolador ESP32 y el servidor remoto, permitiendo la captura y transferencia de imágenes de manera confiable. Además, se busca demostrar la viabilidad de utilizar el ESP32 como una solución integral para aplicaciones de captura de imágenes y comunicación en tiempo real.

A lo largo de la implementación del proyecto, se tendrán en cuenta aspectos como la eficiencia energética, la estabilidad y la escalabilidad del sistema. También se explorarán posibles mejoras y consideraciones adicionales, como el procesamiento de imágenes en el servidor y la integración con técnicas de inteligencia artificial para análisis de imágenes.

En resumen, este proyecto tiene como objetivo desarrollar un sistema basado en el microcontrolador ESP32 para la captura y transferencia de imágenes a través de HTTP. Se espera que este sistema pueda aplicarse en diferentes escenarios, como vigilancia, automatización del hogar, control de accesos, entre otros, brindando una solución confiable y versátil para la captura y transmisión de imágenes en tiempo real.

Es importante destacar que los objetivos y alcances del proyecto pueden ajustarse durante su desarrollo en función de las necesidades y limitaciones identificadas a lo largo del proceso.
